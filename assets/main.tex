\documentclass{article}
\usepackage[utf8]{inputenc}
\usepackage{enumitem}

\title{accelMath Questions: Discrete Math 1}
\author{ionwynsean }
\date{March 2017}

\begin{document}

\maketitle

\section{Rule of Sum and Product}

\begin{enumerate}
    \item A palindrome is a number that reads the same forwards and backwards (like 747). Which of the following numbers is a palindrome?
    
        \begin{itemize}
            \item 100
            \item 101
            \item 102
            \item All of the above
        \end{itemize}
        
    \item How many 3-digit palindromes are there? (Note: The first digit of a 3-digit number must be nonzero.)
    
    \begin{itemize}
        \item Fewer than 100
        \item Exactly 100
        \item More than 100
    \end{itemize}
    
    \item If $A$ is the number of 3-digit palindromes and $B$ is the number of 4-digit palindromes, which is greater?
    
    \begin{itemize}
        \item $A$
        \item $B$
        \item $A$ and $B$ are equal
    \end{itemize}
    
    Solution:
    First let's compute LaTeX: \(A\), the number of 3-digit palindromes. Imagine constructing a 3-digit palindrome by choosing the digits sequentially, starting from the left. There are 9 choices for the first (leftmost) digit, since it can be any digit other than 0. After the first digit has been chosen, there are 10 choices for the second digit (regardless of the choice for the first digit). In order for the number to be a palindrome, the last digit must be the same as the first digit, so after the first digit has been chosen, there is only 1 choice for the last digit. The total number of ways to choose the digits is the product of these numbers:
    LaTeX: \[A = 9 \times 10 \times 1 = 90\]
    Now let's compute LaTeX: \(B\), the number of 4-digit palindromes. Again, imagine constructing a 4-digit palindrome by choosing the digits sequentially, starting from the left. There are 9 choices for the first (leftmost) digit, since it can be any digit other than 0. After the first digit has been chosen, there are 10 choices for the second digit (regardless of the choice for the first digit). In order for the number to be a palindrome, the third digit must be the same as the second digit and the last digit must be the same as the first digit, so after the first and second digits have been chosen, there is only 1 choice for the third digit and 1 choice for the last digit. The total number of ways to choose the digits is the product of these numbers:
    LaTeX: \[B = 9 \times 10 \times 1 \times 1 = 90\]
    Thus, LaTeX: \(A = B\).
    
    \item How many 3-digit palindromes are there where the middle digit is larger than the outer digits (like 121)?
    
    \begin{itemize}
        \item Fewer than 45
        \item Exactly 45
        \item More than 45
    \end{itemize}
    
    \item How many 3-digit palindromes are there where the middle digit is the same as the leading digit?
    
    \begin{itemize}
        \item Fewer than 10
        \item Exactly 10
        \item More than 10
    \end{itemize}

\end{enumerate}

\end{document}

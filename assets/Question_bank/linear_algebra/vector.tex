\documentclass{article}
\usepackage[utf8]{inputenc}
\usepackage{enumitem}
\newcommand\tab[1][1cm]{\hspace*{#1}}
\usepackage{amsmath}

\title{Vectors and Linear Equation}
\author{Hsuan-Yi Lin }
\date{March 2017}

\begin{document}

\maketitle

\section{Vectors and Linear Equation}
\begin{enumerate}
    \item Compute the dot product of u . v \\
        u = (3, 1, 4) and v = (2, 2, 4)
        
         \begin{itemize}
            \item (5, 3, 8)
            \item (6, 2, 16)
            \item (8, 8)
            \item (12, 16)
        \end{itemize}
        
        Answer: the dot product is (3*2, 1*2, 4*4) = (6, 2, 16)
    
    \item the set of all points such that \\
    x = x0 + tv,\\
    represents the line through x0 that is parallel to v and the varaible t is called a parameter.
    Question: what if x0 = 0
    
        \begin{itemize}
            \item plane including origin
            \item a point at the origin 
            \item line passing through the origin
        \end{itemize}    
    
        Answer: it is a line passing thorugh the origin
        
    \item Match the properties of vectors \\
    a) u + v = v + u \\
    b) (u + v) + w = u + (v + w) \\
    c) u + 0 = u \\
    1) zero identity \\ 
    2) associative law \\
    3) commutative law 
        
        \begin{itemize}
            \item a) = 2) , b) = 1), c) = 3)
            \item a) = 3), b) = 2), c) = 1)
            \item a) = 3), b) = 1), c) = 2)
            \item a) = 2), b) = 3), c) = 1)
        \end{itemize}    
        
        Answer: \\
        a) = 3) commutative law \\
        b) = 1) zero identity \\
        c) = 2) associative law 
    
    \item Find the augmented matrix for the linear system\\
        x1 = 1 \\
        x2 = 2 \\
        x3 = 3
        
        \begin{itemize}
            \item $X = \left( \begin{bmatrix} 1& 0& 0&1 \\ 0& 1& 0&2\\ 0& 0& 1&3 \end{bmatrix}\right)$
            \item $Y = \left( \begin{bmatrix} 1& 0& 0 \\ 0& 2& 0\\ 0& 0& 3 \end{bmatrix}\right)$
            \item $Z = \left( \begin{bmatrix} 1& 0& 0 \\ 0& 1& 0\\ 0& 0& 1 \end{bmatrix}\right)$
        \end{itemize}    
        
        Answer: $X = \left( \begin{bmatrix} 1& 0& 0&1 \\ 0& 1& 0&2\\ 0& 0& 1&3 \end{bmatrix}\right)$
    
    \item Fill in the blank. \\
    If a linear system has [  ] solution, its called [ ]

        \begin{itemize}
            \item zero, consistent
            \item at least one, consistent
            \item at least one, inconsistent
            \item infinite, in consistent
        \end{itemize} 
        
        Answer: Theorem states that if a linear system has at least one solution, it's called consistent, otherwise it's call inconsistent
    
    \item What does a linear equation in (a) $R^{2}$ and (b) $R^{3}$ represents?
    
        \begin{itemize}
            \item plane, line
            \item line, plane
            \item dot, line
            \item line, dot
        \end{itemize} 
        
        Answer: a linear equation in (a) $R^{2}$ represents a line and (b) $R^{3}$ represents a plane
    
\end{enumerate}
\end{document}



